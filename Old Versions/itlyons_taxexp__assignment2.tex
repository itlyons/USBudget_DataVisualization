\documentclass[]{article}
\usepackage{lmodern}
\usepackage{amssymb,amsmath}
\usepackage{ifxetex,ifluatex}
\usepackage{fixltx2e} % provides \textsubscript
\ifnum 0\ifxetex 1\fi\ifluatex 1\fi=0 % if pdftex
  \usepackage[T1]{fontenc}
  \usepackage[utf8]{inputenc}
\else % if luatex or xelatex
  \ifxetex
    \usepackage{mathspec}
  \else
    \usepackage{fontspec}
  \fi
  \defaultfontfeatures{Ligatures=TeX,Scale=MatchLowercase}
\fi
% use upquote if available, for straight quotes in verbatim environments
\IfFileExists{upquote.sty}{\usepackage{upquote}}{}
% use microtype if available
\IfFileExists{microtype.sty}{%
\usepackage{microtype}
\UseMicrotypeSet[protrusion]{basicmath} % disable protrusion for tt fonts
}{}
\usepackage[margin=1in]{geometry}
\usepackage{hyperref}
\hypersetup{unicode=true,
            pdftitle={Intro to ggplot2 - Ian Lyons},
            pdfauthor={Ian Lyons},
            pdfborder={0 0 0},
            breaklinks=true}
\urlstyle{same}  % don't use monospace font for urls
\usepackage{graphicx,grffile}
\makeatletter
\def\maxwidth{\ifdim\Gin@nat@width>\linewidth\linewidth\else\Gin@nat@width\fi}
\def\maxheight{\ifdim\Gin@nat@height>\textheight\textheight\else\Gin@nat@height\fi}
\makeatother
% Scale images if necessary, so that they will not overflow the page
% margins by default, and it is still possible to overwrite the defaults
% using explicit options in \includegraphics[width, height, ...]{}
\setkeys{Gin}{width=\maxwidth,height=\maxheight,keepaspectratio}
\IfFileExists{parskip.sty}{%
\usepackage{parskip}
}{% else
\setlength{\parindent}{0pt}
\setlength{\parskip}{6pt plus 2pt minus 1pt}
}
\setlength{\emergencystretch}{3em}  % prevent overfull lines
\providecommand{\tightlist}{%
  \setlength{\itemsep}{0pt}\setlength{\parskip}{0pt}}
\setcounter{secnumdepth}{0}
% Redefines (sub)paragraphs to behave more like sections
\ifx\paragraph\undefined\else
\let\oldparagraph\paragraph
\renewcommand{\paragraph}[1]{\oldparagraph{#1}\mbox{}}
\fi
\ifx\subparagraph\undefined\else
\let\oldsubparagraph\subparagraph
\renewcommand{\subparagraph}[1]{\oldsubparagraph{#1}\mbox{}}
\fi

%%% Use protect on footnotes to avoid problems with footnotes in titles
\let\rmarkdownfootnote\footnote%
\def\footnote{\protect\rmarkdownfootnote}

%%% Change title format to be more compact
\usepackage{titling}

% Create subtitle command for use in maketitle
\newcommand{\subtitle}[1]{
  \posttitle{
    \begin{center}\large#1\end{center}
    }
}

\setlength{\droptitle}{-2em}

  \title{Intro to ggplot2 - Ian Lyons}
    \pretitle{\vspace{\droptitle}\centering\huge}
  \posttitle{\par}
    \author{Ian Lyons}
    \preauthor{\centering\large\emph}
  \postauthor{\par}
      \predate{\centering\large\emph}
  \postdate{\par}
    \date{1/20/2019}


\begin{document}
\maketitle

\hypertarget{read-in-tax-expenditure-data-for-plot-1}{%
\section{Read in tax expenditure data for Plot
1}\label{read-in-tax-expenditure-data-for-plot-1}}

Source for tax expenditures on the Treasury Department Website:
\url{https://home.treasury.gov/policy-issues/tax-policy/tax-expenditures}

\begin{verbatim}
base_df <- read_csv(file = 'tax_expenditures_18to28_renamed.csv')

budg_2019 <- read_csv(file = 'outlays-fy2019.csv')
\end{verbatim}

\hypertarget{reshape-data-for-plot-1}{%
\subsection{Reshape Data For Plot 1}\label{reshape-data-for-plot-1}}

Create a long-form version of the tax expenditures dataset to use in a
time-series plot. That is, create one row for each category, detail, and
fiscal year. Then drop the cumulative `FY2019\_to\_2028' column since I
won't be using it.

\begin{verbatim}
tax_long <- gather(base_df, fiscal_year, expenditure, '2018':'2028', factor_key=TRUE)
tax_long$FY2019_to_2028 <- NULL
\end{verbatim}

\hypertarget{plot-1}{%
\subsection{Plot 1}\label{plot-1}}

\begin{verbatim}
## Create a time-series dataset so that tax expenditures can be compared over the fiscal years included.  

time_series <- tax_long %>% select(fiscal_year, Category, expenditure) %>%
     group_by(fiscal_year, Category) %>%
     tally(expenditure)

## Convert the expenditure column to be in billions of dollars
time_series <- mutate(.data = time_series, amount_billions = n/1000)
time_series$n <- NULL
bar_to_plot <- ggplot(data=time_series, aes(x=fiscal_year, y=amount_billions, color=Category)) + geom_col()

plot1 <- bar_to_plot + 
    scale_y_continuous(name = "Tax Expenditures in Billions of Nominal Dollars", labels =c('$0', '$433', '$866', '$1,300', '$1,733', '$2,166', '$2,600') 
                       , breaks=c(0, 433, 866, 1300, 1733, 2166, 2600), limits = c(0,2600)) + 
    
    scale_x_discrete(name='Fiscal Year') + 
    
    labs(title = "US Federal Government Tax Expenditures, 2018-2028", 
         subtitle = "The United States spends trillions of dollars on tax loopholes each year",
         caption = "Source: United States. Department of Treasury. Tax Policy: Tax Expenditures \n 
         * These estimates are made relative to current law as of July 1, 2018", 
         x = "Fiscal Year" + 
             
             theme(plot.title = element_text(color="black", size=14, face="bold", hjust = 0.5),
                   plot.subtitle = element_text(color="black", size=12, hjust =0.5),
                   axis.title.x = element_text(color="black", size=10),
                   axis.title.y = element_text(color="black", size=10),
                   plot.caption = element_text(color="black", size=8, face="italic")
                   )
    )
\end{verbatim}

\n\n\n

\hypertarget{plot-2-read-in-federal-budget-outlays-data}{%
\section{Plot 2: Read in Federal Budget Outlays
data}\label{plot-2-read-in-federal-budget-outlays-data}}

Source for federal outlays on the Office of Management and Budget
website: * \url{https://www.whitehouse.gov/omb/supplemental-materials/}

\begin{verbatim}
budg_2019 <- read_csv(file = 'outlays-fy2019.csv')
\end{verbatim}

\hypertarget{reshape-data-for-plot-2}{%
\subsection{Reshape Data For Plot 2}\label{reshape-data-for-plot-2}}

Create a long-form version of the outlays dataset to use in a
time-series plot. That is, create one row for each line item (agency
name, bureau name, account name, etc.) and fiscal year.

\begin{verbatim}

budget_longform <- gather(budg_2019, fiscal_year, expenditure, '1962':'2018', factor_key=TRUE)

## Keep only descriptive column names rather than codes. 
keep_cols <- c('Agency Name', 'Bureau Name', 'Account Name', 'Treasury Agency Code', 'Subfunction Title', 'BEA Category', 'Grant/non-grant split' , 'On- or Off- Budget', 'fiscal_year', 'expenditure')

budget_longform <- budget_longform[keep_cols]
\end{verbatim}

\hypertarget{plot-2-group-by-agency-name-and-sum-expenditures-within-the-same-agency-and-year.}{%
\subsection{Plot 2: Group by agency name and sum expenditures within the
same agency and
year.}\label{plot-2-group-by-agency-name-and-sum-expenditures-within-the-same-agency-and-year.}}

\begin{verbatim}
outlays <- budget_longform %>% select(fiscal_year, `Agency Name`, expenditure) %>%
            group_by(fiscal_year, `Agency Name`) %>%
            tally(expenditure)

## Give the expenditures column a descriptive name          
outlays <- mutate(.data = outlays, dollars_thousands = n)
outlays$n <- NULL

## Rename `Agency Name` in snake case.
colnames(outlays)[2] <- 'agency_name'
\end{verbatim}

\hypertarget{plot-2-plot-the-thing}{%
\subsection{Plot 2: Plot the thing!}\label{plot-2-plot-the-thing}}

\begin{verbatim}
ggplot(outlays, aes(x=fiscal_year, y=0)) + geom_point()
\end{verbatim}


\end{document}
